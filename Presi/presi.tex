%----------------------------------------------------------------------------------------
%	PACKAGES AND THEMES
%----------------------------------------------------------------------------------------

\documentclass{beamer}

\mode<presentation> {

%\usetheme{Berlin}
%\usetheme{CambridgeUS}
%\usetheme{Frankfurt}
%\usetheme{Madrid}
%\usetheme{Singapore}
\usetheme{Dresden}

%\usetheme{default}
%\usetheme{AnnArbor}
%\usetheme{Antibes}
%\usetheme{Bergen}
%\usetheme{Berkeley}
%\usetheme{Boadilla}
%\usetheme{Copenhagen}
%\usetheme{Darmstadt}
%\usetheme{Goettingen}
%\usetheme{Hannover}
%\usetheme{Ilmenau}
%\usetheme{JuanLesPins}
%\usetheme{Luebeck}
%\usetheme{Malmoe}
%\usetheme{Marburg}
%\usetheme{Montpellier}
%\usetheme{PaloAlto}
%\usetheme{Pittsburgh}
%\usetheme{Rochester}
%\usetheme{Szeged}
%\usetheme{Warsaw}

% Colors

%\usecolortheme{albatross}
\usecolortheme{beaver}
%\usecolortheme{beetle}
%\usecolortheme{crane}
%\usecolortheme{dolphin}
%\usecolortheme{dove}
%\usecolortheme{fly}
%\usecolortheme{lily}
%\usecolortheme{orchid}
%\usecolortheme{rose}
%\usecolortheme{seagull}
%\usecolortheme{seahorse}
%\usecolortheme{whale}
%\usecolortheme{wolverine}

%\setbeamertemplate{footline} % To remove the footer line in all slides uncomment this line
%\setbeamertemplate{footline}[page number] % To replace the footer line in all slides with a simple slide count uncomment this line

\setbeamertemplate{navigation symbols}{} % To remove the navigation symbols from the bottom of all slides uncomment this line
}

\usepackage[german,ngerman]{babel}
\usepackage[utf8]{inputenc}
\usepackage[T1]{fontenc}
\usepackage{lmodern}
\usepackage{amssymb}
\usepackage{mathtools}
\usepackage{amsmath}
\usepackage{minted}

\usepackage{graphicx} % Allows including images
\usepackage{booktabs} % Allows the use of \toprule, \midrule and \bottomrule in tables

%----------------------------------------------------------------------------------------
%	TITLE PAGE
%----------------------------------------------------------------------------------------

\title[PAPO 2015]{Praktikumsprojekt} 
\author{Oliver Heidmann, Tronje Krabbe}
\institute[UHH]
{
Uni Hamburg \\
Praktikum Parallele Programmierung\\
}
\date{03.06.2015}

\begin{document}

\begin{frame}
\titlepage
\end{frame}

\begin{frame}
\frametitle{Übersicht}
\tableofcontents
\end{frame}

%\AtBeginSection{
%\begin{frame}
%\frametitle{Overview} 
%\tableofcontents[currentsection]
%\end{frame}}

%----------------------------------------------------------------------------------------
%	PRESENTATION SLIDES
%----------------------------------------------------------------------------------------

%------------------------------------------------
\section{Kurzbeschreibung}
%------------------------------------------------

\subsection{}

%------------------------------------------------

\begin{frame}
    \frametitle{Simulation eines Sonnensystems}
    \begin{itemize}
        \item Wir betrachten \textit{n} Objekte
            mit jeweils einer Masse \textit{m}
            und einem Geschwindigkeitsvektor \textit{v}.
        \item Im Zentrum befindet sich ein stationäres Objekt
            mit besonders hoher Masse, analog zu einer Sonne
            oder einem schwarzen Loch.
        \item Ziel: eine vielzahl möglicher stellarer
            Konstellationen simulieren.
    \end{itemize}
\end{frame}

%------------------------------------------------

\begin{frame}
    \frametitle{Funktionsweise}
    Pro Iteration:
    \begin{itemize}
        \item Verrechnung aller Geschwindigkeitsvektoren
        \item Errechnung aktualisierter Positionen
        \item Kollisionsbehandlung (wie?)
    \end{itemize}
\end{frame}

%------------------------------------------------

\begin{frame}
    \frametitle{Schwierigkeiten/Herausforderungen}
    \begin{itemize}
        \item Wie Kollisionen behandeln?
            \begin{itemize}
                \item naiv: ignorieren
                \item beide Objekte zerstören und neue(s) Objekt(e) erzeugen
            \end{itemize}
        \item Größe des Systems
            \begin{itemize}
                \item dynamische Größe? Eher nicht.
                \item Objekte könnten das System verlassen
            \end{itemize}
        \item Visualisierung
    \end{itemize}
\end{frame}

%------------------------------------------------

\section{Lösungsansatz}
\subsection{}

%------------------------------------------------

\begin{frame}[fragile]
    \frametitle{Code}
    \begin{minted}[linenos,
                   samepage,
                   fontsize=\tiny,
                   tabsize=2]{cpp}
class Objects
{
    sortedlist<vector_3D<double>> position;
    sortedlist<vector_3D<double>> velocity;
    sortedlist<double> radius;
    sortedlist<double> mass;

    void add(vector_3D<double> position, vector_3D<double>  velo, double r, double m)
    {
        position.add(position)
        velocity.add(velo)
        radius.add(r)
        mass.add(m)
    }
    vector<Object_data>  calculate_collision(vector<unsigned long>)
    {
        /*
        hier wird berechnet, ob mehrere Objekte mit der Gesamtmasse der
        kollidierenden Objekte gebildet werden,
        oder ob nur ein neues bei der Kollision entsteht.
        Außerdem: Größe, Radius, Geschwindigkeitsvektor.
        */
    }
}
    \end{minted}
\end{frame}

%------------------------------------------------

\section{Projektplan}
\subsection{}

%------------------------------------------------

\begin{frame}
    \frametitle{Projektplan}
    Einmal wöchentlich treffen bis es fertig ist. :)
\end{frame}

%------------------------------------------------

\section{Parallelisierungsschema}
\subsection{}

%------------------------------------------------

\begin{frame}
    \frametitle{Parallelisierung}
    \begin{enumerate}
        \item Naiv: Aufteilung nach Index (jeder Prozess bekommt $n/p$ Objekte)
        \item Cool(?): Aufteilung nach Anzahl beeinflussten Objekten
        \begin{itemize}
            \item Ein Objekt hat Einfluss auf unterschiedlich viele andere Objekte.
                Je nachdem, wie viele beeinflusst werden, muss mehr berechnet werden.
                Verteilt man Objekte nach dieser Gewichtung, kann die Last auf Prozesse
                optimiert werden.
        \end{itemize}
    \end{enumerate}
\end{frame}

%----------------------------------------------------------------------------------------

\end{document} 
